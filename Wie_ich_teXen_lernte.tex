\documentclass[11pt]{scrartcl}
 
 
\usepackage[utf8]{inputenc}
\usepackage{listings}
\begin{document}
\title{\underline{Tex-Doku}}
\maketitle
\section{GIT Versionierung}
Im Folgenden soll erklärt werden, wie wir das mit der GIT-Versionierung gemacht haben.
Zuerst wird per Rechtsklick auf den entsprechenden Ordner die Git-Bash geöffnet:

Neuanlegen eines lokalen Repositorys

Hinzufügen von was Neuem

Committen (-a commitet auch alte hinzugefügte Sachen)

gibt Status

gibt die Historie

gibt die unterschiede (Dokumentenname hinzufuegen)
\begin{lstlisting}
git init
git add dateiname
git commit
git status
git log
git diff
\end{lstlisting}

wichtig ist auch gitignore:
\begin{lstlisting}
vim .gitignore
\end{lstlisting}
Hier zum Beispiel die Dateiendungen reinschreiben, die ignoriert werden sollen, oder Files die ignoriert werden sollen...
\section{Umgang mit vim}
Vim ist der Standardeditor in deiner Konsole. Wichtig: Zuerst i drücken, damit der Insert-Modus reingeht. Nach dem Schreiben mit Escape in den Befehlsmodus wechseln. Mit :wq speichert man und verlässt das ganze.
\section{PDF-File erstellen}
einfach auf den grünen Pfeil klicken. Oder den entsprechenden Befehl (pdflatex) in die Kommandozeile eingeben.

\end{document}
