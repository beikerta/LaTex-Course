% Alexandra Beikert, León-Alexander Hühn, Leon Patzig
% pdfLaTeX
\documentclass{scrartcl}

\usepackage[ngerman]{babel}
\usepackage[utf8]{inputenc}
\usepackage[T1]{fontenc}

\usepackage{pgfplots}
\pgfplotsset{compat=1.16}



\usepackage{pgfplotstable}
\usepackage{gnuplottex}




\begin{document}



\begin{tikzpicture}
\begin{axis}[title=Zerfallsprozess, xlabel={Zeit [s]}, ylabel=Zerfälle]
\addplot+[only marks,
mark=.,
error bars/.cd,
y dir=both, y explicit,]
table
[x=zeit, y=zerfaelle, y error=zerfaelle_err]
{06_messwerte.dat};

\IfFileExists{fit.par} {
\addplot [no markers, red, raw gnuplot] gnuplot { % "raw gnuplot" lässt uns direkt auf gnuplot zugreifen
            f(x) = a*exp(b*x)+c; % Definiere die Funktion zum fitten
            a=system('awk ''{print $1}'' fit.par');
            b=system('awk ''{print $2}'' fit.par'); %Lese parameter ein (Setzt UNIX System bzw System mit awk voraus)
            c=system('awk ''{print $3}'' fit.par');
            plot [x=0:600] f(x); % Wähle Plotgröße, plotte
};
}{
\addplot [no markers, red, raw gnuplot] gnuplot { % "raw gnuplot" lässt uns direkt auf gnuplot zugreifen
           f(x) = a*exp(b*x)+c; % Definiere die Funktion zum fitten
           b=0.00001; % Setze b auf einen kleinen Wert, um floating-point infinity am Anfang und somit NaN im Algorithmus zu vermeiden
           fit f(x) '06_messwerte.dat' using 1:2:3 yerrors via a,b,c; % Wähle die Funktion, Messdaten und Variablen
           plot [x=0:600] f(x); % Wähle Plotgröße, plotte
           set print 'fit.par';
           print a,b,c;
};}


\end{axis}
\end{tikzpicture}


\end{document}

