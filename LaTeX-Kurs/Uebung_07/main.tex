% !TEX program = xelatex
% !TEX encoding = UTF-8 Unicode
% !TEX spellcheck = de_DE

\documentclass{scrreprt}

% !TEX root=../main.tex
% !TEX program = xelatex
% !TEX encoding = UTF-8 Unicode
% !TEX spellcheck = de_DE

\usepackage{blindtext,booktabs,color,nicefrac,polyglossia,setspace,xltxtra,yfonts,hyperref,}

\setmainlanguage{german}

\setromanfont[Mapping=tex-text]{Linux Libertine O}
\setsansfont[Mapping=tex-text]{Linux Biolinum O}

\onehalfspacing


\begin{document}
% Vorgabe: Leerseite
\quad\thispagestyle{empty}\newpage
% https://www.informatik.uni-heidelberg.de/images/perSemester/Formulare/Leitfaden-zur-Erstellung-der-Bachelorarbeit.pdf
% In diesem Leitfaden ist unter Anlage 1 (S. 6) eine Vorlage für die Titelseite enthalten.
% Da jedoch in selbigem Leitfaden geboten wird, »auf ein vernünftiges Layout« (S. 3) zu achten, betrachten wir
% Anlage 1 nur als Orientierungshilfe, alle wichtigen Informationen auf die Titelseite zu bringen.
% Auch halten wir es nicht für sinnvoll, Name, Matrikelnummer, Betreuer und Datum der Abgabe mit diesen Begriffen und
% einem vorangestellten Doppelpunkt in der vorgeschlagenen Form aufzulisten.
% Im Beispiel erhält auch der Titel der Bachelorarbeit viel zu wenig Aufmerksamkeit, weshalb hier eine größere
% Schriftgröße gewählt wird.
\titlehead{Ruprecht-Karls-Universität Heidelberg\\
Fakultät für Mathematik und Informatik\\
Institut für Informatik}
\subject{keine Bachelorarbeit}
\title{\Large
	Messbarkeit quantentypografischer Phänomene in Bachelorarbeiten\\
	Measurability of Quantum Typographical Phenomena in Bachelor Theses
}
\author{Hinz U. Kunz\\
\small Matrikelnummer: 5000042}
\publishers{nicht betreut durch Prof.\,Dr.\,max.\,mus. Termann}
\date{2. November 2048}
\extratitle{\centering Quantentypografie in Bachelorarbeiten}

\maketitle
\tableofcontents

\begin{abstract}
Dieses Dokument dient der Übung des Satzes von umfangreichen Projekten in \LaTeX\index{LaTeX@\LaTeX}. Es gehört zum siebten Übungszettel des \LaTeX-Kurses im Wintersemester 2018\,/\,19. Inhaltlich hat es so ziemlich nichts zu bieten, es könnte aber interessant sein, sich den zugehörigen Sourcecode mal anzusehen, da er eine Menge interessanter \LaTeX-Kommandos enthält.
\end{abstract}

% !TEX root=../main.tex
% !TEX program = xelatex
% !TEX encoding = UTF-8 Unicode
% !TEX spellcheck = de_DE

\setchapterpreamble[o]{\dictum[W. Busch]{Stets findet Überraschung statt. Da, wo man's nicht erwartet hat.}}

\chapter{Einleitung}

\blindtext$\sin(x)\cdot\cos(x) = -\nicefrac{1}{2} \cos(2x)$\footnote{Man beachte auch, dass $\sin(x\pm y) = \sin(x)\cos(y) \pm \cos(x)\sin(y)$}

\begin{table}[h]
  \centering
  \begin{tabular}{ccc}
    \toprule
    a & b & c\\
    d & e & f\\
    g & h & i\\
    \bottomrule
  \end{tabular}
  \caption{Die erste Tabelle}
  \label{tab:tabelle1}
\end{table}
% !TEX root=../main.tex
% !TEX program = xelatex
% !TEX encoding = UTF-8 Unicode
% !TEX spellcheck = de_DE
\setchapterpreamble[o]{\dictum[Cato]{\textsc{Nullus est liber tam malus, ut non aliqua parte prosit.}}}
\index{prosit}


\index{Blinddokument|(}
\blinddocument
\index{Blinddokument|)}
\begin{listing}[H]
\begin{minted}{python}
import numpy as np
x = np.array([9,8,7]) # Beispiel
\end{minted}
\caption{Beispielcode.}
\label{lst:example}
\end{listing}


\begin{figure}[h]
  \centering
  \fbox{I am a picture!}
  \caption{Ein Bild, das die Aussage des Textes unterstreicht.}
  \label{fig:statement}
\end{figure}

% !TEX root=../main.tex
% !TEX program = xelatex
% !TEX encoding = UTF-8 Unicode
% !TEX spellcheck = de_DE
\setchapterpreamble[o]{\dictum[U.\,R. Heber]{Ein schlauer Spruch bereichtert den Kapitelanfang.}}

\chapter{Ein weiteres Kapitel}

\begin{figure}[h]
  \fbox{I am a picture, introducing this chapter!}
  \caption{Bildunterschrift}
  \label{fig:introduction}
\end{figure}

\index{Blindtext}
\Blindtext\footnote{Man beachte auch, dass $\sin(x\pm y) = \sin(x)\cos(y) \pm \cos(x)\sin(y)$}

\begin{table}[h]
  \centering
  \begin{tabular}{ccc}
  \toprule
  eins & zwei & drei\\
  vier & fünf & sechs\\
  sieben & acht & neun\\
  \bottomrule
  \end{tabular}
  \caption{Eine Tabelle mit neun Einträgen}
  \label{tab:tabelle3}
\end{table}

\begin{listing}[H]
\begin{minted}{python}
z = np.ones([6,6,6])
zz=z[::3, 1:4, :]

print(zz)
\end{minted}
\caption{Noch ein Codeschnipsel.}
\label{lst:example}
\end{listing}
% !TEX root=../main.tex
% !TEX program = xelatex
% !TEX encoding = UTF-8 Unicode
% !TEX spellcheck = de_DE
\setchapterpreamble[o]{\dictum[F. Halm]{\hspace*{2em}\textfrak{Ruhe bleibt den Leichen;\\ Der Lebende tauch' frisch ins: Lebens:meer.}}}


\chapter{Und noch ein weiteres Kapitel}

\begin{figure}[p]
  \centering
  \fbox{I am a picture!}
  \caption{Beispiel zu diesem Kapitel}
  \label{fig:example}
\end{figure}

Der \verb$\Bindtext$-Befehl\index{Blindtext} ist eine nette Sache, wenn man in \LaTeX\index{LaTeX@\LaTeX} sehen will, wie ein Dokument mit viel Inhalt aussieht, ohne, dass man Inhalt hat. \\
\index{Blindtext}
\Blindtext

\begin{figure}[p]
  \centering
  \fbox{I am a picture!}
  \caption{Veranschaulichung der Aussage}
  \label{fig:illustration}
\end{figure}

\begin{figure}[p]
  \centering
  \fbox{I am a picture!}
  \caption{Detailansicht}
  \label{fig:detail}
\end{figure}

\begin{figure}[p]
  \centering
  \fbox{I am a picture!}
  \caption{Visualisierung des Ergebnisses}
  \label{fig:visualization}
\end{figure}

%\appendix
\listoffigures
\listoftables

\end{document}
