% !TEX root=../main.tex
% !TEX program = xelatex
% !TEX encoding = UTF-8 Unicode
% !TEX spellcheck = de_DE
\setchapterpreamble[o]{\dictum[U.\,R. Heber]{Ein schlauer Spruch bereichtert den Kapitelanfang.}}

\chapter{Ein weiteres Kapitel}

\begin{figure}[h]
  \fbox{I am a picture, introducing this chapter!}
  \caption{Bildunterschrift}
  \label{fig:introduction}
\end{figure}

\index{Blindtext}
\Blindtext\footnote{Man beachte auch, dass $\sin(x\pm y) = \sin(x)\cos(y) \pm \cos(x)\sin(y)$}

\begin{table}[h]
  \centering
  \begin{tabular}{ccc}
  \toprule
  eins & zwei & drei\\
  vier & fünf & sechs\\
  sieben & acht & neun\\
  \bottomrule
  \end{tabular}
  \caption{Eine Tabelle mit neun Einträgen}
  \label{tab:tabelle3}
\end{table}

\begin{listing}[H]
\begin{minted}{python}
z = np.ones([6,6,6])
zz=z[::3, 1:4, :]

print(zz)
\end{minted}
\caption{Noch ein Codeschnipsel.}
\label{lst:example}
\end{listing}