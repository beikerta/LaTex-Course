\documentclass{scrartcl}

\begin{document}
\section{Sitzung 1: 15.10.18}
2 ECTS Punkte, Montags ist Uebungszettelausgabe, fuer Physiker benotet. Die Abgabe geht ueber abgabe@latexkurs.de. TeX ist eine Markup Language, das bedeutet, dass man bestimmte Kommandos braucht, um etwas einzuleiten. Diese Kommandos beginnen mit einem Backslash. \\
Ganz am Anfang steht immer ein documentclass\{TYPE\}. Typische Typen sind article, report, scrartcl(article-Erweiterung), beamer(Präsentationen),... \\
Mit usepackage kann man Pakete einbinden, die zur Arbeitserleichterung dienen. \\
Mit den folgenden Befehlen kann man Schriftgroessen veraendern:\\
{\large gross} {\Large  groesser} {\LARGE  groesser} {\Huge am groessten} {\normalsize  und normal.} \\
Beim kompilieren der TeX-Datei wird neben der .pdf auch eine .log, eine .aux und eventuell andere Hilfsdateien erzeugt.
Die Voraussetzung fuer diesen Kurs ist TeX Live 2018.


\end{document}