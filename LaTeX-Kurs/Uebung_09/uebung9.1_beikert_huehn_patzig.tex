% Alexandra Beikert, León-Alexander Hühn, Leon Patzig
% XeLaTeX
\documentclass{beamer}
\usepackage{polyglossia}
\setmainlanguage[spelling=new, babelshorthands=true]{german}
\usepackage{amsmath}
\usepackage{showexpl}
\usetheme{CambridgeUS}
\usecolortheme{seagull}
\beamertemplatenavigationsymbolsempty
\title{Matrizen}
\author{Beikert, Hühn, Patzig}
\begin{document}
\frame{\titlepage}
\begin{frame}{Matrizen in \LaTeX}{Die Anleitung}
\tableofcontents
\end{frame}
\section{Typen von Matrizen}
\begin{frame}[fragile]{Matrizen in AMSmath}{Typen von Matrizen}
\begin{itemize}
\item Es gibt insgesamt 6 Matrixumgebungen im amsmath Packet
	\begin{itemize}
	\item<+-> \texttt{matrix}
	\item<+->  \texttt{pmatrix}
	\item<+->  \texttt{bmatrix}
	\item<+->  \texttt{Bmatrix}
	\item<+->  \texttt{vmatrix}
	\item<+->  \texttt{Vmatrix}
	\end{itemize}
\item Die Umgebungen haben verschiedene Klammerstile
\item Sie müssen im Mathemodus gesetzt werden
\item<+->  Mit \verb|\dots|-Befehlen können verschiedene Punkte gesetzt werden
\end{itemize}
\end{frame}
\section{Die Umgebungen}
\begin{frame}[fragile]{Die Umgebungen}{Verschiedene Klammerstile}
\begin{itemize}
\item<+-> Die Standartumgebung \texttt{matrix} setzt keine Klammern
	\begin{itemize}
	\item<+->  Mit \verb|\left| und \verb|\right| können eigene Klammern gesetzt werden
	\end{itemize}
\item<+->  \texttt{pmatrix} Setzt runde Klammern
	\begin{itemize}
	\item<+->  Auch für Vektoren geeignet
	\end{itemize}
\item<+->  \texttt{bmatrix} setzt eckige Klammern, \texttt{Bmatrix} geschweifte Klammern
\item<+->  \texttt{vmatrix} setzt gerade Striche, \texttt{Vmatrix} zwei gerade Striche
\end{itemize}
\end{frame}
\section{Ein Beispiel}
\begin{frame}[fragile]{Ein Beispiel}
\begin{LTXexample}
% In der Präämbel
\usepackage{amsmath}
% Im Dokument
\[
\begin{pmatrix}
a & b & c & \hdots & z \\
b & c & \hdots & z & a\\
\vdots &&\ddots&& \vdots \\
z & a & \hdotsfor{3}
\end{pmatrix}
\]
\end{LTXexample}
\end{frame}
\end{document}