%Alexandra Beikert, León-Alexander Hühn, Leon Patzig
%pdfLaTeX
\documentclass{scrartcl}

\usepackage[spanish,french,ngerman]{babel}
\usepackage[T1]{fontenc}
\usepackage[utf8]{inputenc}




\begin{document}
Ich studiere zwar keine Sprache, noch spreche ich eine Sprache, welche nicht das lateinische Alphabet verwendet - die griechischen Buchstaben in mathematischen Formeln einmal ausgenommen. Allerdings habe ich eine Freundin, welche Latein, Griechisch und Englisch studiert. Mit babel kann sie sehr leicht im Text zwischen den Sprachen wechseln.

\selectlanguage{french}
Ceci est mon premier texte que j'ai écrit avec \LaTeX  en français. Il est un peu ennuyeux que vous devez entrer une commande à chaque changement de langue. Mais c'est normal avec le \LaTeX.

\selectlanguage{spanish}
Si usa un idioma que no usa el alfabeto latino, la mejor opción es la poliglosia.

\selectlanguage{ngerman}
Aber für die hier gebräuchlichsten Sprachen, welche alle das lateinische Alphabet verwenden, reicht babel völlig aus und ist nahezu identisch mit polyglossia.

\selectlanguage{french}
Je recommande LaTeX à mon amie. Quel paquet elle utilise est pour elle alors peu importe. Quelle serait l'alternative? Microsoft Office Word?


\end{document}
