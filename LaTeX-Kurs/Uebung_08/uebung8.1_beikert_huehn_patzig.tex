%Alexandra Beikert, León-Alexander Hühn, Leon Patzig
%XeLaTeX
\documentclass{scrartcl}

\usepackage{polyglossia}
\setmainlanguage{german}

\usepackage[style=authoryear]{biblatex}
\usepackage{hyperref}

\addbibresource{library.bib}

\begin{document}

Für die meisten Physikstudenten ist das Hinzusiehen von Büchern zusätzlich zum Skript einer Vorlesung unerlässlich. \\
So ist für Vorlesungen der theoretischen Physik das Standartwerk der ``Bartelmann'' (\cite{Bartelmann}).
Befindet man sich hingegen in der misslichen Lage, das sog. PAP durchführen zu müssen und kann sich partout nicht an die relevanten Inhalte erinnern, sollte man einen Blick in den ``Gerthsen'' (\cite{Gerthsen}) werfen.
Sollte man sich für die Astronomy als Wahlpflicht entschieden haben, so wird man sich im Moment wohl mit der Sternenentstehung beschäftigen. Prof. Klessen empfielt ein Buch von \cite{astro}.
Muss man sich wieder einmal mit Lebesgue-Integralen herumärgern, kann man einen Blick in \cite[][Kap. X]{aescher} werfen.\\
Ein Fakt am Ende: Heinrich Hertz sagte bereits 1887 den Photoeffekt vorraus.\footnote{\cite[vgl.][S. 998f.]{Hertz1887}}

\newpage

\printbibliography[title=Bücher, type=book]
\printbibliography[title=Artikel, type=article]

\end{document}
